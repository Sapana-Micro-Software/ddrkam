\documentclass[12pt]{article}
\usepackage{amsmath}
\usepackage{amssymb}
\usepackage{graphicx}
\usepackage{hyperref}

\title{Data-Driven Hierarchical Runge-Kutta and Adams Methods\\for Nonlinear Dynamical Systems}
\author{Shyamal Suhana Chandra}
\date{2025}

\begin{document}

\maketitle

\begin{abstract}
This paper presents a comprehensive implementation of the Runge-Kutta 3rd order method and Adams methods for solving nonlinear differential equations. We introduce a novel data-driven hierarchical architecture inspired by transformer networks that enhances traditional numerical integration methods. The framework is implemented in C/C++ with Objective-C visualization capabilities, making it suitable for macOS and VisionOS platforms.
\end{abstract}

\section{Introduction}

Numerical methods for solving ordinary differential equations (ODEs) are fundamental tools in scientific computing. The Runge-Kutta family of methods, particularly the 3rd order variant, provides a good balance between accuracy and computational efficiency.

\section{Runge-Kutta 3rd Order Method}

The Runge-Kutta 3rd order method (RK3) is defined by the following stages:

\begin{align}
k_1 &= f(t_n, y_n) \\
k_2 &= f(t_n + \frac{h}{2}, y_n + \frac{h}{2}k_1) \\
k_3 &= f(t_n + h, y_n - hk_1 + 2hk_2) \\
y_{n+1} &= y_n + \frac{h}{6}(k_1 + 4k_2 + k_3)
\end{align}

where $h$ is the step size, $f$ is the ODE function, and $y_n$ is the state at time $t_n$.

\section{Adams Methods}

Adams-Bashforth and Adams-Moulton methods are multi-step methods that use information from previous steps.

\subsection{Adams-Bashforth 3rd Order}

The predictor step:
\begin{equation}
y_{n+1} = y_n + \frac{h}{12}(23f_n - 16f_{n-1} + 5f_{n-2})
\end{equation}

\subsection{Adams-Moulton 3rd Order}

The corrector step:
\begin{equation}
y_{n+1} = y_n + \frac{h}{12}(5f_{n+1} + 8f_n - f_{n-1})
\end{equation}

\section{Hierarchical Data-Driven Architecture}

We propose a hierarchical architecture inspired by transformer networks that processes ODE solutions through multiple layers with attention mechanisms. Each layer applies transformations to the state space, enabling adaptive refinement of the numerical solution.

The hierarchical solver consists of:
\begin{itemize}
\item Multiple processing layers with learnable weights
\item Attention mechanisms for state-space transformations
\item Adaptive step size control based on hierarchical features
\end{itemize}

\section{Implementation}

The framework is implemented in C/C++ for core numerical methods, with Objective-C wrappers for visualization and integration with Apple platforms.

\section{Results}

[Results section would contain experimental validation]

\section{Conclusion}

We have presented a comprehensive framework for solving nonlinear ODEs using traditional and data-driven hierarchical methods, suitable for deployment on Apple platforms.

\bibliographystyle{plain}
\begin{thebibliography}{9}
\bibitem{rk3}
Butcher, J. C. (2008). \textit{Numerical Methods for Ordinary Differential Equations}. Wiley.

\bibitem{adams}
Gear, C. W. (1971). \textit{Numerical Initial Value Problems in Ordinary Differential Equations}. Prentice-Hall.
\end{thebibliography}

\end{document}
