\documentclass[12pt]{article}
\usepackage[margin=0.5in]{geometry}
\usepackage{amsmath}
\usepackage{listings}
\usepackage{xcolor}
\usepackage{hyperref}
\usepackage{booktabs}
\usepackage{longtable}
\usepackage{array}
\newcolumntype{C}[1]{>{\centering\arraybackslash}p{#1}}

\title{DDRKAM Reference Manual\\Data-Driven Runge-Kutta and Adams Methods}
\author{Shyamal Suhana Chandra}
\date{2025}

\lstset{
    language=C,
    basicstyle=\ttfamily\small,
    keywordstyle=\color{blue},
    commentstyle=\color{green},
    stringstyle=\color{red},
    numbers=left,
    numberstyle=\tiny,
    breaklines=true
}

\begin{document}

\maketitle

\tableofcontents
\newpage

\section{Introduction}

This manual provides comprehensive documentation for the DDRKAM (Data-Driven Runge-Kutta and Adams Methods) framework. The framework implements numerical methods for solving ordinary differential equations (ODEs) with support for traditional and hierarchical data-driven approaches.

The framework includes:
\begin{itemize}
\item Euler's Method (1st order)
\item Data-Driven Euler's Method (DDEuler)
\item Runge-Kutta 3rd Order Method (RK3)
\item Data-Driven Runge-Kutta 3rd Order (DDRK3)
\item Adams Methods (AM)
\item Data-Driven Adams Methods (DDAM)
\end{itemize}

\section{Euler's Method}

\subsection{Overview}

Euler's Method is the simplest numerical method for solving ODEs. It is a first-order explicit method with local truncation error $O(h^2)$.

\subsection{Algorithm}

\begin{equation}
y_{n+1} = y_n + h \cdot f(t_n, y_n)
\end{equation}

where $h$ is the step size, $f$ is the ODE function, and $y_n$ is the state at time $t_n$.

\subsection{API Reference}

\subsubsection{euler\_step}

Performs a single integration step using Euler's method.

\begin{lstlisting}
double euler_step(ODEFunction f, double t0, double* y0, 
                  size_t n, double h, void* params);
\end{lstlisting}

\textbf{Parameters:}
\begin{itemize}
\item \texttt{f}: Function pointer to the ODE system
\item \texttt{t0}: Current time
\item \texttt{y0}: Current state vector (modified in-place)
\item \texttt{n}: Dimension of the system
\item \texttt{h}: Step size
\item \texttt{params}: User-defined parameters
\end{itemize}

\textbf{Returns:} New time value (t0 + h)

\subsubsection{euler\_solve}

Solves an ODE system over a time interval using Euler's method.

\begin{lstlisting}
size_t euler_solve(ODEFunction f, double t0, double t_end, 
                   const double* y0, size_t n, double h, 
                   void* params, double* t_out, double* y_out);
\end{lstlisting}

\section{Data-Driven Euler's Method}

\subsection{Overview}

Data-Driven Euler's Method (DDEuler) extends standard Euler's method with a hierarchical transformer-inspired architecture that applies adaptive corrections to improve accuracy.

\subsection{Algorithm}

\begin{equation}
y_{n+1} = y_n + h \cdot f(t_n, y_n) + h \cdot \alpha \cdot \text{Attention}(y_n)
\end{equation}

where $\alpha$ is a learning rate and $\text{Attention}(y_n)$ is computed through hierarchical transformer layers.

\subsection{API Reference}

\subsubsection{hierarchical\_euler\_init}

Initializes a Data-Driven Euler solver.

\begin{lstlisting}
int hierarchical_euler_init(HierarchicalEulerSolver* solver, 
                            size_t num_layers, size_t state_dim, 
                            size_t hidden_dim);
\end{lstlisting}

\subsubsection{hierarchical\_euler\_step}

Performs a single integration step using Data-Driven Euler.

\begin{lstlisting}
double hierarchical_euler_step(HierarchicalEulerSolver* solver, 
                               ODEFunction f, double t, double* y, 
                               double h, void* params);
\end{lstlisting}

\subsubsection{hierarchical\_euler\_solve}

Solves an ODE system using Data-Driven Euler over a time interval.

\begin{lstlisting}
size_t hierarchical_euler_solve(HierarchicalEulerSolver* solver, 
                                 ODEFunction f, double t0, double t_end, 
                                 const double* y0, double h, void* params, 
                                 double* t_out, double* y_out);
\end{lstlisting}

\section{Parallel and Distributed Methods}

\subsection{Overview}

All methods support parallel, distributed, concurrent, hierarchical, and stacked execution modes. This enables:
\begin{itemize}
\item Multi-threaded execution (OpenMP, pthreads)
\item Distributed computing (MPI)
\item Concurrent execution of multiple methods
\item Hierarchical/stacked architectures
\item Enhanced performance and scalability
\end{itemize}

\subsection{Parallel Runge-Kutta}

\subsubsection{parallel\_rk\_init}

Initialize parallel RK3 solver.

\begin{lstlisting}
int parallel_rk_init(ParallelRKSolver* solver, size_t state_dim,
                     size_t num_workers, ParallelMode mode,
                     StackedConfig* stacked);
\end{lstlisting}

\subsubsection{parallel\_rk\_step}

Perform parallel RK3 step.

\begin{lstlisting}
double parallel_rk_step(ParallelRKSolver* solver, ODEFunction f,
                        double t, double* y, double h, void* params);
\end{lstlisting}

\subsubsection{stacked\_rk\_step}

Perform stacked/hierarchical RK3 step.

\begin{lstlisting}
double stacked_rk_step(ParallelRKSolver* solver, ODEFunction f,
                      double t, double* y, double h, void* params);
\end{lstlisting}

\subsubsection{concurrent\_rk\_execute}

Execute multiple RK3 instances concurrently.

\begin{lstlisting}
int concurrent_rk_execute(ParallelRKSolver* solvers[], size_t num_solvers,
                          ODEFunction f, double t, const double* y, double h,
                          void* params, double** results);
\end{lstlisting}

\section{Real-Time, Online, and Dynamic Methods}

\subsection{Real-Time Methods}

Real-time methods process streaming data with minimal latency.

\subsubsection{realtime\_rk\_init}

Initialize real-time RK3 solver.

\begin{lstlisting}
int realtime_rk_init(RealtimeRKSolver* solver, size_t state_dim, 
                     double step_size, DataCallback callback, 
                     void* callback_data);
\end{lstlisting}

\subsubsection{realtime\_rk\_step}

Perform real-time RK3 step with streaming support.

\begin{lstlisting}
double realtime_rk_step(RealtimeRKSolver* solver, ODEFunction f,
                        double t, double* y, double h, void* params);
\end{lstlisting}

\subsection{Online Methods}

Online methods adapt to incoming data with incremental learning.

\subsubsection{online\_rk\_init}

Initialize online RK3 solver.

\begin{lstlisting}
int online_rk_init(OnlineRKSolver* solver, size_t state_dim,
                   double initial_step_size, double learning_rate);
\end{lstlisting}

\subsubsection{online\_rk\_step}

Perform online RK3 step with adaptive step size.

\begin{lstlisting}
double online_rk_step(OnlineRKSolver* solver, ODEFunction f,
                     double t, double* y, void* params);
\end{lstlisting}

\subsection{Dynamic Methods}

Dynamic methods provide fully adaptive execution.

\subsubsection{dynamic\_rk\_init}

Initialize dynamic RK3 solver.

\begin{lstlisting}
int dynamic_rk_init(DynamicRKSolver* solver, size_t state_dim,
                    double initial_step_size, double adaptation_rate);
\end{lstlisting}

\subsubsection{dynamic\_rk\_step}

Perform dynamic RK3 step with adaptive parameters.

\begin{lstlisting}
double dynamic_rk_step(DynamicRKSolver* solver, ODEFunction f,
                       double t, double* y, void* params);
\end{lstlisting}

\section{Nonlinear Programming Solvers}

\subsection{Karmarkar's Algorithm}

Karmarkar's Algorithm is a polynomial-time interior point method for linear programming. It provides guaranteed polynomial-time convergence for linear programming problems.

\subsubsection{karmarkar\_solver\_init}

Initialize Karmarkar solver.

\begin{lstlisting}
int karmarkar_solver_init(KarmarkarSolver* solver, size_t state_dim,
                          ADAMSolverType type, double alpha, double beta,
                          double mu, double epsilon, const double* c,
                          const double** A, const double* b,
                          size_t num_constraints);
\end{lstlisting}

\subsubsection{karmarkar\_ode\_solve}

Solve ODE using Karmarkar's algorithm.

\begin{lstlisting}
int karmarkar_ode_solve(KarmarkarSolver* solver, ODEFunction f,
                        double t0, double t_end, const double* y0,
                        void* params, double* y_out);
\end{lstlisting}

\subsection{Nonlinear ODE Solver}

\subsubsection{nonlinear\_ode\_init}

Initialize nonlinear ODE solver using NLP methods.

\begin{lstlisting}
int nonlinear_ode_init(NonlinearODESolver* solver, size_t state_dim,
                       NLPSolverType solver_type, ObjectiveFunction objective,
                       ConstraintFunction constraints, void* params);
\end{lstlisting}

\subsubsection{nonlinear\_ode\_solve}

Solve ODE using nonlinear programming.

\begin{lstlisting}
int nonlinear_ode_solve(NonlinearODESolver* solver, ODEFunction f,
                       double t0, double t_end, const double* y0,
                       double* y_out);
\end{lstlisting}

\subsection{Nonlinear PDE Solver}

\subsubsection{nonlinear\_pde\_init}

Initialize nonlinear PDE solver.

\begin{lstlisting}
int nonlinear_pde_init(NonlinearPDESolver* solver, size_t spatial_dim,
                      const size_t* grid_size, NLPSolverType solver_type,
                      PDEFunction pde_func, void* params);
\end{lstlisting}

\subsubsection{nonlinear\_pde\_solve}

Solve PDE using nonlinear programming.

\begin{lstlisting}
int nonlinear_pde_solve(NonlinearPDESolver* solver, double t0, double t_end,
                      const double* u0, double* u_out);
\end{lstlisting}

\section{Additional Distributed, Data-Driven, Online, Real-Time Solvers}

\subsection{Distributed Data-Driven Solver}

Combines distributed computing with data-driven methods.

\subsection{Online Data-Driven Solver}

Combines online learning with data-driven methods.

\subsection{Real-Time Data-Driven Solver}

Combines real-time processing with data-driven methods.

\subsection{Distributed Online Solver}

Combines distributed computing with online learning.

\subsection{Distributed Real-Time Solver}

Combines distributed computing with real-time processing.

\section{Runge-Kutta 3rd Order Method}

\subsection{Overview}

The Runge-Kutta 3rd order method provides a good balance between accuracy and computational efficiency for solving ODEs.

\subsection{API Reference}

\subsubsection{rk3\_step}

Performs a single integration step using RK3.

\begin{lstlisting}
double rk3_step(ODEFunction f, double t0, double* y0, 
                size_t n, double h, void* params);
\end{lstlisting}

\textbf{Parameters:}
\begin{itemize}
\item \texttt{f}: Function pointer to the ODE system
\item \texttt{t0}: Current time
\item \texttt{y0}: Current state vector (modified in-place)
\item \texttt{n}: Dimension of the system
\item \texttt{h}: Step size
\item \texttt{params}: User-defined parameters
\end{itemize}

\textbf{Returns:} New time value (t0 + h)

\subsubsection{rk3\_solve}

Solves an ODE system over a time interval.

\begin{lstlisting}
size_t rk3_solve(ODEFunction f, double t0, double t_end, 
                 const double* y0, size_t n, double h, 
                 void* params, double* t_out, double* y_out);
\end{lstlisting}

\textbf{Parameters:}
\begin{itemize}
\item \texttt{f}: Function pointer to the ODE system
\item \texttt{t0}: Initial time
\item \texttt{t\_end}: Final time
\item \texttt{y0}: Initial state vector
\item \texttt{n}: Dimension of the system
\item \texttt{h}: Step size
\item \texttt{params}: User-defined parameters
\item \texttt{t\_out}: Output time array (allocated by caller)
\item \texttt{y\_out}: Output state array (n $\times$ num\_steps, allocated by caller)
\end{itemize}

\textbf{Returns:} Number of steps taken

\subsection{Example}

\begin{lstlisting}
void lorenz(double t, const double* y, double* dydt, void* params) {
    double* p = (double*)params;
    double sigma = p[0], rho = p[1], beta = p[2];
    dydt[0] = sigma * (y[1] - y[0]);
    dydt[1] = y[0] * (rho - y[2]) - y[1];
    dydt[2] = y[0] * y[1] - beta * y[2];
}

double params[3] = {10.0, 28.0, 8.0/3.0};
double y0[3] = {1.0, 1.0, 1.0};
double t_out[100];
double y_out[300];
size_t steps = rk3_solve(lorenz, 0.0, 1.0, y0, 3, 0.01, 
                         params, t_out, y_out);
\end{lstlisting}

\section{Adams Methods}

\subsection{Adams-Bashforth 3rd Order}

Predictor step for multi-step integration.

\begin{lstlisting}
void adams_bashforth3(ODEFunction f, const double* t, 
                      const double* y, size_t n, double h, 
                      void* params, double* y_pred);
\end{lstlisting}

\subsection{Adams-Moulton 3rd Order}

Corrector step for multi-step integration.

\begin{lstlisting}
void adams_moulton3(ODEFunction f, const double* t, 
                    const double* y, size_t n, double h, 
                    void* params, const double* y_pred, 
                    double* y_corr);
\end{lstlisting}

\section{Hierarchical Runge-Kutta Method}

\subsection{Overview}

The hierarchical RK method uses a transformer-like architecture with multiple processing layers and attention mechanisms.

\subsection{API Reference}

\subsubsection{hierarchical\_rk\_init}

Initializes a hierarchical RK solver.

\begin{lstlisting}
int hierarchical_rk_init(HierarchicalRKSolver* solver, 
                         size_t num_layers, size_t state_dim, 
                         size_t hidden_dim);
\end{lstlisting}

\textbf{Returns:} 0 on success, -1 on failure

\subsubsection{hierarchical\_rk\_free}

Frees resources allocated by the solver.

\begin{lstlisting}
void hierarchical_rk_free(HierarchicalRKSolver* solver);
\end{lstlisting}

\subsubsection{hierarchical\_rk\_solve}

Solves an ODE using the hierarchical method.

\begin{lstlisting}
size_t hierarchical_rk_solve(HierarchicalRKSolver* solver,
                             ODEFunction f, double t0, double t_end,
                             const double* y0, double h, void* params,
                             double* t_out, double* y_out);
\end{lstlisting}

\section{Objective-C Framework}

\subsection{DDRKAMSolver}

Main solver class for Objective-C applications.

\begin{lstlisting}
DDRKAMSolver* solver = [[DDRKAMSolver alloc] 
                         initWithDimension:3];
NSDictionary* result = [solver solveWithFunction:^(double t, 
                                                    const double* y, 
                                                    double* dydt, 
                                                    void* params) {
    // ODE definition
} startTime:0.0 endTime:1.0 
initialState:@[@1.0, @1.0, @1.0] 
stepSize:0.01 params:NULL];
\end{lstlisting}

\subsection{DDRKAMVisualizer}

Visualization component for plotting solutions.

\begin{lstlisting}
DDRKAMVisualizer* viz = [[DDRKAMVisualizer alloc] init];
NSView* view = [viz createVisualizationViewWithTime:timeArray
                                              state:stateArray
                                          dimension:3];
[viz exportToCSV:@"/path/to/output.csv" 
            time:timeArray 
           state:stateArray];
\end{lstlisting}

\subsection{DDRKAMHierarchicalSolver}

Hierarchical solver for Objective-C.

\begin{lstlisting}
DDRKAMHierarchicalSolver* solver = 
    [[DDRKAMHierarchicalSolver alloc] 
     initWithDimension:3 numLayers:4 hiddenDim:32];
\end{lstlisting}

\section{Map/Reduce Framework}

\subsection{Overview}

The Map/Reduce framework provides distributed ODE solving on commodity hardware with fault tolerance through redundancy. It partitions the state space across mapper nodes, processes derivatives in parallel, and aggregates results through reducer nodes.

\subsection{API Reference}

\subsubsection{mapreduce\_ode\_init}

Initializes a Map/Reduce ODE solver.

\begin{lstlisting}
int mapreduce_ode_init(MapReduceODESolver* solver, 
                       size_t state_dim,
                       const MapReduceConfig* config);
\end{lstlisting}

\textbf{Parameters:}
\begin{itemize}
\item \texttt{solver}: Pointer to Map/Reduce solver structure
\item \texttt{state\_dim}: Dimension of the ODE system
\item \texttt{config}: Configuration structure with mapper/reducer counts, redundancy settings, etc.
\end{itemize}

\textbf{Returns:} 0 on success, -1 on failure

\subsubsection{mapreduce\_ode\_solve}

Solves an ODE system using Map/Reduce framework.

\begin{lstlisting}
int mapreduce_ode_solve(MapReduceODESolver* solver,
                        ODEFunction f,
                        double t0, double t_end,
                        const double* y0,
                        double h, void* params,
                        double* y_out);
\end{lstlisting}

\textbf{Parameters:}
\begin{itemize}
\item \texttt{solver}: Initialized Map/Reduce solver
\item \texttt{f}: ODE function pointer
\item \texttt{t0, t\_end}: Time interval
\item \texttt{y0}: Initial state
\item \texttt{h}: Step size
\item \texttt{params}: User-defined parameters
\item \texttt{y\_out}: Output state
\end{itemize}

\textbf{Returns:} 0 on success, -1 on failure

\subsubsection{mapreduce\_estimate\_cost}

Estimates the computational cost for Map/Reduce execution.

\begin{lstlisting}
double mapreduce_estimate_cost(const MapReduceODESolver* solver,
                                double* compute_hours,
                                double* network_cost);
\end{lstlisting}

\subsection{Example}

\begin{lstlisting}
#include "mapreduce_solvers.h"

MapReduceODESolver solver;
MapReduceConfig config = {
    .num_mappers = 4,
    .num_reducers = 2,
    .chunk_size = 100,
    .enable_redundancy = 1,
    .redundancy_factor = 3,
    .use_commodity_hardware = 1,
    .network_bandwidth = 100.0,
    .compute_cost_per_hour = 0.10
};

mapreduce_ode_init(&solver, 1000, &config);

double y0[1000] = {1.0, ...};
double y_out[1000];
mapreduce_ode_solve(&solver, my_ode, 0.0, 1.0, y0, 0.01, NULL, y_out);

double cost = mapreduce_estimate_cost(&solver, NULL, NULL);
mapreduce_ode_free(&solver);
\end{lstlisting}

\section{Apache Spark Framework}

\subsection{Overview}

The Apache Spark framework provides distributed ODE solving using Resilient Distributed Datasets (RDDs) for fault-tolerant computation. Spark offers superior performance for iterative algorithms through RDD caching and lineage-based recovery.

\subsection{API Reference}

\subsubsection{spark\_ode\_init}

Initializes a Spark ODE solver.

\begin{lstlisting}
int spark_ode_init(SparkODESolver* solver,
                   size_t state_dim,
                   const SparkConfig* config);
\end{lstlisting}

\textbf{Parameters:}
\begin{itemize}
\item \texttt{solver}: Pointer to Spark solver structure
\item \texttt{state\_dim}: Dimension of the ODE system
\item \texttt{config}: Configuration structure with executor counts, caching settings, etc.
\end{itemize}

\textbf{Returns:} 0 on success, -1 on failure

\subsubsection{spark\_ode\_solve}

Solves an ODE system using Spark framework.

\begin{lstlisting}
int spark_ode_solve(SparkODESolver* solver,
                    ODEFunction f,
                    double t0, double t_end,
                    const double* y0,
                    double h, void* params,
                    double* y_out);
\end{lstlisting}

\textbf{Parameters:}
\begin{itemize}
\item \texttt{solver}: Initialized Spark solver
\item \texttt{f}: ODE function pointer
\item \texttt{t0, t\_end}: Time interval
\item \texttt{y0}: Initial state
\item \texttt{h}: Step size
\item \texttt{params}: User-defined parameters
\item \texttt{y\_out}: Output state
\end{itemize}

\textbf{Returns:} 0 on success, -1 on failure

\subsubsection{spark\_estimate\_cost}

Estimates the computational cost for Spark execution.

\begin{lstlisting}
double spark_estimate_cost(const SparkODESolver* solver,
                           double* compute_hours,
                           double* network_cost,
                           double* storage_cost);
\end{lstlisting}

\subsection{Example}

\begin{lstlisting}
#include "spark_solvers.h"

SparkODESolver solver;
SparkConfig config = {
    .num_executors = 4,
    .cores_per_executor = 2,
    .memory_per_executor = 2048,
    .num_partitions = 8,
    .enable_caching = 1,
    .enable_checkpointing = 1,
    .checkpoint_interval = 1.0,
    .use_commodity_hardware = 1,
    .network_bandwidth = 100.0,
    .compute_cost_per_hour = 0.10,
    .enable_dynamic_allocation = 1
};

spark_ode_init(&solver, 1000, &config);

double y0[1000] = {1.0, ...};
double y_out[1000];
spark_ode_solve(&solver, my_ode, 0.0, 1.0, y0, 0.01, NULL, y_out);

double cost = spark_estimate_cost(&solver, NULL, NULL, NULL);
spark_ode_free(&solver);
\end{lstlisting}

\section{Non-Orthodox Architectures}

\subsection{Micro-Gas Jet Circuit}

Micro-gas jet circuits use fluid dynamics for computation. See \texttt{nonorthodox\_architectures.h} for API.

\subsection{Dataflow (Arvind)}

Tagged token dataflow computing for fine-grained parallelism.

\subsection{ACE (Turing)}

Turing's stored-program computer architecture implementation.

\subsection{Systolic Array}

Regular array of processing elements with local communication.

\subsection{TPU (Patterson)}

Google TPU architecture for matrix acceleration.

\subsection{GPU Architectures}

Support for CUDA, Metal, Vulkan, and AMD GPU acceleration.

\subsection{Spiralizer with Chord Algorithm (Chandra, Shyamal)}

Spiralizer architecture combining Chord distributed hash tables with Robert Morris collision hashing (MIT) and spiral traversal patterns.

\textbf{API:}
\begin{lstlisting}[language=C]
SpiralizerChordConfig config = {
    .num_nodes = 256,
    .finger_table_size = 8,
    .hash_table_size = 1024,
    .enable_morris_hashing = 1,
    .enable_spiral_traversal = 1
};
SpiralizerChordSolver solver;
spiralizer_chord_ode_init(&solver, n, &config);
spiralizer_chord_ode_solve(&solver, f, t0, t_end, y0, h, params, y_out);
spiralizer_chord_ode_free(&solver);
\end{lstlisting}

\subsection{Lattice Architecture (Waterfront variation - Chandra, Shyamal)}

Variation of Turing's Waterfront architecture, presented by USC alum from HP Labs at MIT event online at Strata. Multi-dimensional lattice with Waterfront buffering.

\textbf{API:}
\begin{lstlisting}[language=C]
LatticeWaterfrontConfig config = {
    .lattice_dimensions = 4,
    .nodes_per_dimension = 16,
    .waterfront_size = 256,
    .enable_waterfront_buffering = 1,
    .enable_lattice_routing = 1
};
LatticeWaterfrontSolver solver;
lattice_waterfront_ode_init(&solver, n, &config);
lattice_waterfront_ode_solve(&solver, f, t0, t_end, y0, h, params, y_out);
lattice_waterfront_ode_free(&solver);
\end{lstlisting}

\subsection{Standard Parallel Computing Architectures}

\textbf{MPI (Message Passing Interface):}
\begin{lstlisting}[language=C]
MPIConfig config = {
    .num_processes = 8,
    .process_rank = 0,
    .communication_buffer_size = 1024,
    .enable_collective_ops = 1
};
MPISolver solver;
mpi_ode_init(&solver, n, &config);
mpi_ode_solve(&solver, f, t0, t_end, y0, h, params, y_out);
mpi_ode_free(&solver);
\end{lstlisting}

\textbf{OpenMP (Open Multi-Processing):}
\begin{lstlisting}[language=C]
OpenMPConfig config = {
    .num_threads = 8,
    .chunk_size = 64,
    .schedule_type = 1,  // dynamic
    .enable_affinity = 1
};
OpenMPSolver solver;
openmp_ode_init(&solver, n, &config);
openmp_ode_solve(&solver, f, t0, t_end, y0, h, params, y_out);
openmp_ode_free(&solver);
\end{lstlisting}

\textbf{Pthreads (POSIX Threads):}
\begin{lstlisting}[language=C]
PthreadsConfig config = {
    .num_threads = 8,
    .enable_work_stealing = 1,
    .enable_barrier_sync = 1
};
PthreadsSolver solver;
pthreads_ode_init(&solver, n, &config);
pthreads_ode_solve(&solver, f, t0, t_end, y0, h, params, y_out);
pthreads_ode_free(&solver);
\end{lstlisting}

\subsection{Specialized Hardware Architectures}

\textbf{FPGA AWS F1 (Xilinx UltraScale+):}
\begin{lstlisting}[language=C]
FPGAAWSF1Config config = {
    .num_fpga_devices = 1,
    .num_logic_cells = 2500000,
    .num_dsp_slices = 6840,
    .pcie_bandwidth = 16,  // GB/s
    .enable_hls_acceleration = 1
};
FPGAAWSF1Solver solver;
fpga_aws_f1_ode_init(&solver, n, &config);
fpga_aws_f1_ode_solve(&solver, f, t0, t_end, y0, h, params, y_out);
fpga_aws_f1_ode_free(&solver);
\end{lstlisting}

\textbf{TilePU Sunway (SW26010):}
\begin{lstlisting}[language=C]
TilePUSunwayConfig config = {
    .num_core_groups = 4,
    .cores_per_group = 64,
    .num_management_cores = 4,
    .enable_dma = 1,
    .enable_register_communication = 1
};
TilePUSunwaySolver solver;
tilepu_sunway_ode_init(&solver, n, &config);
tilepu_sunway_ode_solve(&solver, f, t0, t_end, y0, h, params, y_out);
tilepu_sunway_ode_free(&solver);
\end{lstlisting}

\textbf{Coprocessor Intel Xeon Phi:}
\begin{lstlisting}[language=C]
CoprocessorXeonPhiConfig config = {
    .num_cores = 72,
    .num_threads_per_core = 4,
    .high_bandwidth_memory = 16,  // GB
    .enable_wide_vector = 1,  // 512-bit
    .enable_mic_architecture = 1
};
CoprocessorXeonPhiSolver solver;
coprocessor_xeon_phi_ode_init(&solver, n, &config);
coprocessor_xeon_phi_ode_solve(&solver, f, t0, t_end, y0, h, params, y_out);
coprocessor_xeon_phi_ode_free(&solver);
\end{lstlisting}

\subsection{Additional Architectures}

See \texttt{nonorthodox\_architectures.h} for:
\begin{itemize}
\item Standard Parallel Computing: MPI (Message Passing Interface), OpenMP (Open Multi-Processing), Pthreads (POSIX Threads)
\item GPGPU (General-Purpose GPU) for platform-agnostic GPU computing
\item Vector Processor architecture for SIMD data-parallel operations
\item Specialized Hardware: ASIC (Application-Specific Integrated Circuit), FPGA (Field-Programmable Gate Array), FPGA AWS F1 (Xilinx UltraScale+), DSP (Digital Signal Processor)
\item Quantum Processing Units: QPU Azure (Microsoft Quantum), QPU Intel Horse Ridge (cryogenic quantum control)
\item Specialized Processing Units: TilePU Mellanox (Tile-GX72), TilePU Sunway (SW26010), DPU Microsoft (biological computation), MFPU (Microfluidic Processing Unit), NPU (Neuromorphic Processing Unit), LPU Lightmatter (photonic computing)
\item AsAP (Asynchronous Array of Simple Processors) - UC Davis architecture
\item Coprocessor: Intel Xeon Phi many-core coprocessor with wide vector units
\item Massively-Threaded (Korf) - Frontier search with massive threading
\item STARR (Chandra et al.) - Semantic memory architecture - https://github.com/shyamalschandra/STARR
\item TrueNorth (IBM), Loihi (Intel), BrainChips - Neuromorphic architectures
\item Racetrack (Parkin), Phase Change Memory (IBM Research)
\item Lyric (MIT), HW Bayesian Networks (Chandra)
\item Semantic Lexographic Binary Search (Chandra \& Chandra)
\item Kernelized SPS Binary Search (Chandra, Shyamal)
\item Multiple-Search Representation Tree Algorithm
\end{itemize}

\subsection{Multiple-Search Representation Tree Algorithm}

Uses multiple search strategies (BFS, DFS, A*, Best-First) with tree and graph state representations.

\textbf{API:}
\begin{lstlisting}[language=C]
MultipleSearchTreeConfig config = {
    .max_tree_depth = 100,
    .max_nodes = 10000,
    .num_search_strategies = 4,
    .enable_bfs = 1,
    .enable_dfs = 1,
    .enable_astar = 1,
    .enable_best_first = 1,
    .heuristic_weight = 1.0
};
MultipleSearchTreeSolver solver;
multiple_search_tree_ode_init(&solver, n, &config);
multiple_search_tree_ode_solve(&solver, f, t0, t_end, y0, h, params, y_out);
multiple_search_tree_ode_free(&solver);
\end{lstlisting}

\section{Platform Support}

\begin{itemize}
\item macOS 10.13+
\item iOS 11.0+
\item visionOS 1.0+
\end{itemize}

\section{Copyright}

Copyright (C) 2025, Shyamal Suhana Chandra

All rights reserved.

\end{document}
