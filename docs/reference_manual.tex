\documentclass[12pt]{article}
\usepackage{amsmath}
\usepackage{listings}
\usepackage{xcolor}
\usepackage{hyperref}

\title{DDRKAM Reference Manual\\Data-Driven Runge-Kutta and Adams Methods}
\author{Shyamal Suhana Chandra}
\date{2025}

\lstset{
    language=C,
    basicstyle=\ttfamily\small,
    keywordstyle=\color{blue},
    commentstyle=\color{green},
    stringstyle=\color{red},
    numbers=left,
    numberstyle=\tiny,
    breaklines=true
}

\begin{document}

\maketitle

\tableofcontents
\newpage

\section{Introduction}

This manual provides comprehensive documentation for the DDRKAM (Data-Driven Runge-Kutta and Adams Methods) framework. The framework implements numerical methods for solving ordinary differential equations (ODEs) with support for traditional and hierarchical data-driven approaches.

\section{Runge-Kutta 3rd Order Method}

\subsection{Overview}

The Runge-Kutta 3rd order method provides a good balance between accuracy and computational efficiency for solving ODEs.

\subsection{API Reference}

\subsubsection{rk3\_step}

Performs a single integration step using RK3.

\begin{lstlisting}
double rk3_step(ODEFunction f, double t0, double* y0, 
                size_t n, double h, void* params);
\end{lstlisting}

\textbf{Parameters:}
\begin{itemize}
\item \texttt{f}: Function pointer to the ODE system
\item \texttt{t0}: Current time
\item \texttt{y0}: Current state vector (modified in-place)
\item \texttt{n}: Dimension of the system
\item \texttt{h}: Step size
\item \texttt{params}: User-defined parameters
\end{itemize}

\textbf{Returns:} New time value (t0 + h)

\subsubsection{rk3\_solve}

Solves an ODE system over a time interval.

\begin{lstlisting}
size_t rk3_solve(ODEFunction f, double t0, double t_end, 
                 const double* y0, size_t n, double h, 
                 void* params, double* t_out, double* y_out);
\end{lstlisting}

\textbf{Parameters:}
\begin{itemize}
\item \texttt{f}: Function pointer to the ODE system
\item \texttt{t0}: Initial time
\item \texttt{t\_end}: Final time
\item \texttt{y0}: Initial state vector
\item \texttt{n}: Dimension of the system
\item \texttt{h}: Step size
\item \texttt{params}: User-defined parameters
\item \texttt{t\_out}: Output time array (allocated by caller)
\item \texttt{y\_out}: Output state array (n $\times$ num\_steps, allocated by caller)
\end{itemize}

\textbf{Returns:} Number of steps taken

\subsection{Example}

\begin{lstlisting}
void lorenz(double t, const double* y, double* dydt, void* params) {
    double* p = (double*)params;
    double sigma = p[0], rho = p[1], beta = p[2];
    dydt[0] = sigma * (y[1] - y[0]);
    dydt[1] = y[0] * (rho - y[2]) - y[1];
    dydt[2] = y[0] * y[1] - beta * y[2];
}

double params[3] = {10.0, 28.0, 8.0/3.0};
double y0[3] = {1.0, 1.0, 1.0};
double t_out[100];
double y_out[300];
size_t steps = rk3_solve(lorenz, 0.0, 1.0, y0, 3, 0.01, 
                         params, t_out, y_out);
\end{lstlisting}

\section{Adams Methods}

\subsection{Adams-Bashforth 3rd Order}

Predictor step for multi-step integration.

\begin{lstlisting}
void adams_bashforth3(ODEFunction f, const double* t, 
                      const double* y, size_t n, double h, 
                      void* params, double* y_pred);
\end{lstlisting}

\subsection{Adams-Moulton 3rd Order}

Corrector step for multi-step integration.

\begin{lstlisting}
void adams_moulton3(ODEFunction f, const double* t, 
                    const double* y, size_t n, double h, 
                    void* params, const double* y_pred, 
                    double* y_corr);
\end{lstlisting}

\section{Hierarchical Runge-Kutta Method}

\subsection{Overview}

The hierarchical RK method uses a transformer-like architecture with multiple processing layers and attention mechanisms.

\subsection{API Reference}

\subsubsection{hierarchical\_rk\_init}

Initializes a hierarchical RK solver.

\begin{lstlisting}
int hierarchical_rk_init(HierarchicalRKSolver* solver, 
                         size_t num_layers, size_t state_dim, 
                         size_t hidden_dim);
\end{lstlisting}

\textbf{Returns:} 0 on success, -1 on failure

\subsubsection{hierarchical\_rk\_free}

Frees resources allocated by the solver.

\begin{lstlisting}
void hierarchical_rk_free(HierarchicalRKSolver* solver);
\end{lstlisting}

\subsubsection{hierarchical\_rk\_solve}

Solves an ODE using the hierarchical method.

\begin{lstlisting}
size_t hierarchical_rk_solve(HierarchicalRKSolver* solver,
                             ODEFunction f, double t0, double t_end,
                             const double* y0, double h, void* params,
                             double* t_out, double* y_out);
\end{lstlisting}

\section{Objective-C Framework}

\subsection{DDRKAMSolver}

Main solver class for Objective-C applications.

\begin{lstlisting}
DDRKAMSolver* solver = [[DDRKAMSolver alloc] 
                         initWithDimension:3];
NSDictionary* result = [solver solveWithFunction:^(double t, 
                                                    const double* y, 
                                                    double* dydt, 
                                                    void* params) {
    // ODE definition
} startTime:0.0 endTime:1.0 
initialState:@[@1.0, @1.0, @1.0] 
stepSize:0.01 params:NULL];
\end{lstlisting}

\subsection{DDRKAMVisualizer}

Visualization component for plotting solutions.

\begin{lstlisting}
DDRKAMVisualizer* viz = [[DDRKAMVisualizer alloc] init];
NSView* view = [viz createVisualizationViewWithTime:timeArray
                                              state:stateArray
                                          dimension:3];
[viz exportToCSV:@"/path/to/output.csv" 
            time:timeArray 
           state:stateArray];
\end{lstlisting}

\subsection{DDRKAMHierarchicalSolver}

Hierarchical solver for Objective-C.

\begin{lstlisting}
DDRKAMHierarchicalSolver* solver = 
    [[DDRKAMHierarchicalSolver alloc] 
     initWithDimension:3 numLayers:4 hiddenDim:32];
\end{lstlisting}

\section{Platform Support}

\begin{itemize}
\item macOS 10.13+
\item iOS 11.0+
\item visionOS 1.0+
\end{itemize}

\section{Copyright}

Copyright (C) 2025, Shyamal Suhana Chandra

All rights reserved.

\end{document}
